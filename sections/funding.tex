\begin{description}

    \item[2023-2027] \href{https://cnrs-naviterm.github.io/}{NaviTerm project} --
                    %\emph{Navigation terminologique pour une montée en compétence rapide et personnalisée sur un domaine de recherche}
                    \emph{Terminological navigation through the scientific literature} \\
                    %
                    Funded by the French Defense Innovation Agency and the CNRS, 213$k$€, sole PI \\[.1em]
                    %
                    %\emph{Ce projet vise à accélérer la manière dont les chercheurs se familiarisent avec un nouveau domaine en générant automatiquement des représentations navigables des connaissances scientifiques}
                    \emph{This project aims to accelerate how researchers familiarize themselves with a new field by automatically generating navigable representations of scientific knowledge}\\[.1em]
                    %
                    Collaborators: Béatrice Daille (Nantes Université), Richard Dufour (Nantes Université)

    \item[2020-2025] \href{https://anr-delices.github.io/}{DELICES project} --
                    %\emph{Indexer la littérature scientifique par expansion sémantique}
                    \emph{Indexing scientific literature through semantic expansion} \\
                    %
                    Young Researcher project (JCJC) funded by the ANR, 200$k$€, sole PI \\[.1em]
                    %
                    %\emph{Ce projet vise à améliorer l'efficacité de recherche des moteurs dans les bibliothèques numériques scientifiques en enrichissant leur indexation avec des mots-clés générés automatiquement}
                    \emph{This project aims to improve the retrieval effectiveness of search engines in scientific digital libraries by enriching their indexing with automatically generated keyphrases}\\[.1em]
                    %
                    Collaborators: Béatrice Daille (Nantes Université), Evelyne Jacquey (ATILF), Jian-Yun Nie (Université de Montréal, Canada)

    \item[2020] WASP project --
                \emph{Building a writing assistance system for scientific papers} \\
                 %Atlantic 2020 grant for a University of Tokyo researcher visit (canceled due to COVID-19), 10$k$€
                 \href{https://Atlanstic2020.fr/}{Atlanstic2020} grant for a visiting researcher from the University of Tokyo (canceled by COVID-19) \\[.1em]
                 %
                 %\emph{Ce projet vise à concevoir un outil d’aide à l’écriture scientifique, permettant aux chercheurs de sélectionner les expressions figées les plus adaptées à leurs articles}
                 \emph{This project aims to design a writing assistance tool to help researchers select the most appropriate formulaic expressions for their articles} \\[.1em]
                 %
                 Collaborator: Akiko Aizawa (National Institute of Informatics, Japan)
                 
    
    \item[2019] IKEBANA project -- 
                \emph{Improving keyphrase extraction by adopting neural architectures} \\
                \href{https://Atlanstic2020.fr/}{Atlanstic2020} sabbatical grant, research stay at the National Institute of Informatics, 20$k$€ \\[.1em]
                %
                %\emph{Ce projet vise à améliorer l’efficacité des méthodes d’extraction automatique de mots-clés en adoptant des modèles de réseaux de neurones récurrents}
                \emph{This project aims to improve the effectiveness of keyphrase extraction methods by adopting recurrent neural network models} \\[.1em]
                %
                Collaborator: Akiko Aizawa (National Institute of Informatics, Japan)

    \item[2016] \href{http://boudinfl.github.io/talias/}{TALIAS project} -- 
                %\emph{LE TAL au service de l'Indexation des Articles Scientifiques}
                \emph{Natural Language Processing for indexing scientific articles} \\
                %
                PEPS INS2I project funded by the CNRS, 5$k$€ \\[.1em]
                %
                %\emph{Ce projet vise à évaluer la robustesse des modèles d’extraction de mots-clés face à différents niveaux de bruit dans les documents}
                \emph{This project aims to evaluate the robustness of keyphrase extraction models across varying levels of noise in documents}

    \item[2015] \href{http://boudinfl.github.io/golem/}{GOLEM project} -- 
                %\emph{Approche par Optimisation pour l'Extraction de Mots-clés}
                \emph{An Optimization-Based Approach for Keyphrase Extraction} \\
                PEPS INS2I/INSMI project funded by the CNRS, 6$k$€ \\
                %
                Collaborator: Evgeny Gurevsky (Nantes Université) \\[.1em]
                %\emph{Ce projet vise à explorer l’utilisation de méthodes de Programmation Linéaire en Nombres Entiers pour l’extraction automatique de mots-clés}
                \emph{This project aims to explore the use of Integer Linear Programming for automatic keyphrase extraction}
                
\end{description}
